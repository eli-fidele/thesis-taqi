%The \introduction command is provided as a convenience.
%if you want special chapter formatting, you'll probably want to avoid using it altogether

  \chapter*{Introduction}
       \addcontentsline{toc}{chapter}{Introduction}
\chaptermark{Introduction}
\markboth{Introduction}{Introduction}

	% The three lines above are to make sure that the headers are right, that the intro gets included in the table of contents, and that it doesn't get numbered 1 so that chapter one is 1.

So, what are \textit{spectral statistics}? Do they have to do with rainbows? Sceptres? No, they don’t, but they’re almost as colorful and regal. The word spectral is borrowed from the spectral-like patterns observed in statistical physics - whether it may be atomic spectra or other quantum mechanical phenomena. The borrowing is loose and not literal, but still somewhat well founded.

The field of Random Matrix Theory was extensively developed in the 1930s by the nuclear physicist Eugene Wigner. He found connections between the deterministic properties of atomic nuclei and their random and stochastic behaviors. The link? Random matrices.

So in the context of this thesis, \textit{Spectral statistics} will be an umbrella term for random matrix statistics that somehow involve that matrix's eigenvalues and eigenvectors.



\minititle{The RMAT Package}

\begin{displayquote}
Tell Me and I Forget; Teach Me and I May Remember; Involve Me and I Learn.
-Confucius
\end{displayquote}

To explore these spectral statistics, this thesis will use the $\textbf{RMAT}$ package. This package was developed alongside this thesis in order to facilitate the simulation of these random matrices and spectral statistics. As such, there is a large simulation component to this thesis. To showcase the methodology of the simulations, code snippets will be sprinkled about the thesis. It will use code derived from the package RMAT which can be found on GitHub; minimal source code will also be available in the appendix, as well as pre-simulated data available on the Reed database. All code examples in this thesis are reproducible. Simply use the RMAT package (or equivalent code) and use $set.seed(23)$.

You are strongly encouraged to try out these simulations yourself!

Instructions on how to get RMAT. Try CRAN then Github then source code in appendix. The source code was functional as of R 4.0.5.
