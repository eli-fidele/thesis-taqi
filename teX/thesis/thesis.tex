% This is the Reed College LaTeX thesis template. Most of the work 
% for the document class was done by Sam Noble (SN), as well as this
% template. Later comments etc. by Ben Salzberg (BTS). Additional
% restructuring and APA support by Jess Youngberg (JY).
% Your comments and suggestions are more than welcome; please email
% them to cus@reed.edu
%
% See http://web.reed.edu/cis/help/latex.html for help. There are a 
% great bunch of help pages there, with notes on
% getting started, bibtex, etc. Go there and read it if you're not
% already familiar with LaTeX.
%
% Any line that starts with a percent symbol is a comment. 
% They won't show up in the document, and are useful for notes 
% to yourself and explaining commands. 
% Commenting also removes a line from the document; 
% very handy for troubleshooting problems. -BTS

% As far as I know, this follows the requirements laid out in 
% the 2002-2003 Senior Handbook. Ask a librarian to check the 
% document before binding. -SN

%%
%% Preamble
%%
% \documentclass{<something>} must begin each LaTeX document
\documentclass[12pt,twoside]{reedthesis}
% Packages are extensions to the basic LaTeX functions. Whatever you
% want to typeset, there is probably a package out there for it.
% Chemistry (chemtex), screenplays, you name it.
% Check out CTAN to see: http://www.ctan.org/
%%
\usepackage{graphicx,latexsym} 
\usepackage{amssymb,amsthm,amsmath}
\usepackage{longtable,booktabs,setspace} 
\usepackage{chemarr} %% Useful for one reaction arrow, useless if you're not a chem major
\usepackage[hyphens]{url}
\usepackage{rotating}
\usepackage{natbib}
% Comment out the natbib line above and uncomment the following two lines to use the new 
% biblatex-chicago style, for Chicago A. Also make some changes at the end where the 
% bibliography is included. 
%\usepackage{biblatex-chicago}
%\bibliography{thesis}

% \usepackage{times} % other fonts are available like times, bookman, charter, palatino

\title{Eigenanalysis of Erdos-Renyi Graphs}
\author{Ali Taqi}
% The month and year that you submit your FINAL draft TO THE LIBRARY (May or December)
\date{May 2021}
\division{Mathematics and Natural Sciences}
\advisor{Jonathan M. Wells}
%If you have two advisors for some reason, you can use the following
%\altadvisor{Your Other Advisor}
%%% Remember to use the correct department!
\department{Mathematics}
% if you're writing a thesis in an interdisciplinary major,
% uncomment the line below and change the text as appropriate.
% check the Senior Handbook if unsure.
%\thedivisionof{The Established Interdisciplinary Committee for}
% if you want the approval page to say "Approved for the Committee",
% uncomment the next line
%\approvedforthe{Committee}

\setlength{\parskip}{0pt}
%%
%% End Preamble
%%
%% The fun begins:
\begin{document}

  \maketitle
  \frontmatter % this stuff will be roman-numbered
  \pagestyle{empty} % this removes page numbers from the frontmatter

% Acknowledgements (Acceptable American spelling) are optional
% So are Acknowledgments (proper English spelling)
    \chapter*{Acknowledgements}
	I want to thank a few people.

% The preface is optional
% To remove it, comment it out or delete it.
    \chapter*{Preface}
	This is an example of a thesis setup to use the reed thesis document class.
	
	

    \chapter*{List of Abbreviations}
		You can always change the way your abbreviations are formatted. Play around with it yourself, use tables, or come to CUS if you'd like to change the way it looks. You can also completely remove this chapter if you have no need for a list of abbreviations. Here is an example of what this could look like:

	\begin{table}[h]
	\centering % You could remove this to move table to the left
	\begin{tabular}{ll}
		\textbf{ABC}  	&  American Broadcasting Company \\
		\textbf{CBS}  	&  Columbia Broadcasting System\\
		\textbf{CDC}  	&  Center for Disease Control \\
		\textbf{CIA}  	&  Central Intelligence Agency\\
		\textbf{CLBR} 	&  Center for Life Beyond Reed\\
		\textbf{CUS}  	&  Computer User Services\\
		\textbf{FBI}  	&  Federal Bureau of Investigation\\
		\textbf{NBC}  	&  National Broadcasting Corporation\\
	\end{tabular}
	\end{table}
	

    \tableofcontents
% if you want a list of tables, optional
    \listoftables
% if you want a list of figures, also optional
    \listoffigures

% The abstract is not required if you're writing a creative thesis (but aren't they all?)
% If your abstract is longer than a page, there may be a formatting issue.
    \chapter*{Abstract}
	The preface pretty much says it all.
	
	\chapter*{Dedication}
	You can have a dedication here if you wish.

  \mainmatter % here the regular arabic numbering starts
  \pagestyle{fancyplain} % turns page numbering back on

%The \introduction command is provided as a convenience.
%if you want special chapter formatting, you'll probably want to avoid using it altogether

    \chapter*{Introduction}
         \addcontentsline{toc}{chapter}{Introduction}
	\chaptermark{Introduction}
	\markboth{Introduction}{Introduction}
	% The three lines above are to make sure that the headers are right, that the intro gets included in the table of contents, and that it doesn't get numbered 1 so that chapter one is 1.

% Double spacing: if you want to double space, or one and a half 
% space, uncomment one of the following lines. You can go back to 
% single spacing with the \singlespacing command.
% \onehalfspacing
% \doublespacing
	
	Welcome to the \LaTeX\ thesis template. If you've never used \TeX\ or \LaTeX\ before, you'll have an initial learning period to go through, but the results of a nicely formatted thesis are worth it for more than the aesthetic benefit: markup like \LaTeX\ is more consistent than the output of a word processor, much less prone to corruption or crashing and the resulting file is smaller than a Word file. While you may have never had problems using Word in the past, your thesis is going to be about twice as large and complex as anything you've written before, taxing Word's capabilities. If you're still on the fence about  using \LaTeX, read the Introduction to LaTeX on the CUS site as well as skim the following template and give it a few weeks. Pretty soon all the markup gibberish will become second nature.

\section{Why use it?}
	
\LaTeX\ does a great job of formatting tables and paragraphs. Its line-breaking algorithm was the subject of a PhD.\thinspace thesis. It does a fine job of automatically inserting ligatures, and to top it all off it is the only way to typeset good-looking mathematics.

\section{Who should use it?}

Anyone who needs to use math, tables, a lot of figures, complex cross-references, IPA or who just cares about the final appearance of their document should use \LaTeX. At Reed, math majors are required to use it, most physics majors will want to use it, and many other science majors may want it also.
	

%*****************************************************************************************
%   Chapter 1
%*****************************************************************************************

%=========================================================================================
%   D-Distributions
%=========================================================================================
\section{$\D$-distributions}
%=========================================================================================
\begin{frame}
\frametitle{$\D$-distribution}

  \begin{alertblock}{\deftitle{$\D-distribution$}}
    Suppose $P$ is a $\D$-distributed random matrix. Then, we notate this $P \sim \D$. In the simplest of terms, $\D$ is essentially the algorithm that generates the entries of $P$.
    We define two primary methods of distribution: \textbf{explicit} distribution, and \textbf{implicit} distribution.
    If $\D$ is an explicit distribution, then some or all the entries of $P$ are independent random variables with a given distribution.
    Otherwise, if $\D$ is implicit, then the matrix has dependent entries imposed by the algorithm that generates it.
  \end{alertblock}

\end{frame}
%=========================================================================================
\begin{frame} \frametitle{Homogenous Explicit $\D$-distributions}

  \begin{alertblock}{\deftitle{Homogenous Explicit $\D$-distributions}}
    Suppose $P \sim \D$ where $\D$ is a homogenous and explicit distribution. Additionally, let $\D^*$ denote the corresponding random variable analogue of $\D$.
    Then, every single entry of $P$ is an i.i.d random variable with the corresponding distribution. That is,
    $$ P \sim \D \iff \forall i,j \mid p_{ij} \sim \D^* $$
  \end{alertblock}

  \begin{examples}
  Suppose $P \sim \Normal(0,1)$ and that $P$ is a $2 \times 2$ matrix.
  Then, $p_{11}, p_{12}, p_{21}, p_{22}$ are independent, identically distributed random variables with the standard normal distribution.
  \end{examples}

\end{frame}
%=========================================================================================
\begin{frame} \frametitle{Non-Homogenous Explicit $\D$-distributions}

  \begin{alertblock}{\deftitle{Diagonal Bands}}
    Suppose $P = (p_{ij})$ is an $N \times N$ matrix. Then, $P$ may be partitioned into $2n - 1$ rows called diagonal bands. Each band is denoted $[\rho]_P$ where $[\rho]_P = \{p_{ij} \mid \rho = i - j\}$. We have
    $\rho \in \{ -(N-1), \dots, -1, 0, 1, \dots, N-1 \}$.
  \end{alertblock}

\end{frame}
%=========================================================================================
\begin{frame} \frametitle{The Hermite $\beta$-Matrix}

  \begin{alertblock}{\deftitle{$\b$-matrix}}
    Suppose $P \sim \H(\b)$ is an $N \times N$ matrix. Then, the main diagonal $[0]_P \sim \Normal(0,2)$.
    Additionally, both the main off-digaonals are equal and they are given by $[1]_{P} = [-1]_{P} = \vec{X} = (X_k)_{k=1}^{N-1}$ where $X_k \sim \chi(\text{df} = \beta k)$.
    As such, we obtain a Hermite-$\b$ distributed matrix. Note that this is a symmetric tridiagonal matrix.
  \end{alertblock}

\end{frame}
%=========================================================================================
\begin{frame} \frametitle{Implicit Distributions}



\end{frame}
%=========================================================================================


%=========================================================================================
%   D-Distributions
%=========================================================================================
\section{Random Matrices}
%=========================================================================================
\begin{frame} \frametitle{Random Matrices}

\begin{alertblock}{Random Matrices}
Let $P \sim \D$ be an $N \times N$ matrix over $\F$. Then, the entries of $P$ are elements in $\F$ completely determined by the $\D$-distribution, regardless of what type it is.
Also, if $\D$ is an explicit distribution, $\D^\dagger$ represents the symmetric/hermitian version of $\D$.
\end{alertblock}
\end{frame}

%=========================================================================================
\begin{frame} \frametitle{Random Matrix Ensembles}

\begin{alertblock}{\deftitle{Random Matrix Ensembles}}
A $\D$-distributed ensemble $\Ens$ of $N \times N$ random matrices over $\F$ of size $K$ is defined as a set of $K$ iterations of that class of random matrix, and it is denoted:
$$ \Ens = \bigcup_{i = 1}^K P_i \where P_i \sim \mathcal{D} \and P_i \in \F^{N \times N} $$
\end{alertblock}

\end{frame}
%=========================================================================================
\begin{frame} \frametitle{Summary of $\D$-distributions}
  \begin{center}
    \Ddisttable
  \end{center}
\end{frame}

% %=========================================================================================
% \begin{frame}
% \frametitle{Empty Slide}
% \end{frame}
% %=========================================================================================
% \begin{frame}
% \frametitle{Empty Slide}
% \end{frame}


%=========================================================================================
%=========================================================================================


%If you feel it necessary to include an appendix, it goes here.
    \appendix
      \chapter{The First Appendix}
      \chapter{The Second Appendix, for Fun}


%This is where endnotes are supposed to go, if you have them.
%I have no idea how endnotes work with LaTeX.

  \backmatter % backmatter makes the index and bibliography appear properly in the t.o.c...

% if you're using bibtex, the next line forces every entry in the bibtex file to be included
% in your bibliography, regardless of whether or not you've cited it in the thesis.
    \nocite{*}

% Rename my bibliography to be called "Works Cited" and not "References" or ``Bibliography''
% \renewcommand{\bibname}{Works Cited}

%    \bibliographystyle{bsts/mla-good} % there are a variety of styles available; 
%  \bibliographystyle{plainnat}
% replace ``plainnat'' with the style of choice. You can refer to files in the bsts or APA 
% subfolder, e.g. 
 \bibliographystyle{APA/apa-good}  % or
 \bibliography{thesis}
 % Comment the above two lines and uncomment the next line to use biblatex-chicago.
 %\printbibliography[heading=bibintoc]

% Finally, an index would go here... but it is also optional.
\end{document}
