
%*****************************************************************************************
%   Chapter 1
%*****************************************************************************************

%=========================================================================================
%   D-Distributions
%=========================================================================================
\section{$\D$-distributions}
%=========================================================================================
\begin{frame}
\frametitle{$\D$-distribution}

  \begin{alertblock}{\deftitle{$\D-distribution$}}
    Suppose $P$ is a $\D$-distributed random matrix. Then, we notate this $P \sim \D$. In the simplest of terms, $\D$ is essentially the algorithm that generates the entries of $P$.
    We define two primary methods of distribution: \textbf{explicit} distribution, and \textbf{implicit} distribution.
    If $\D$ is an explicit distribution, then some or all the entries of $P$ are independent random variables with a given distribution.
    Otherwise, if $\D$ is implicit, then the matrix has dependent entries imposed by the algorithm that generates it.
  \end{alertblock}

\end{frame}
%=========================================================================================
\begin{frame} \frametitle{Homogenous Explicit $\D$-distributions}

  \begin{alertblock}{\deftitle{Homogenous Explicit $\D$-distributions}}
    Suppose $P \sim \D$ where $\D$ is a homogenous and explicit distribution. Additionally, let $\D^*$ denote the corresponding random variable analogue of $\D$.
    Then, every single entry of $P$ is an i.i.d random variable with the corresponding distribution. That is,
    $$ P \sim \D \iff \forall i,j \mid p_{ij} \sim \D^* $$
  \end{alertblock}

  \begin{examples}
  Suppose $P \sim \Normal(0,1)$ and that $P$ is a $2 \times 2$ matrix.
  Then, $p_{11}, p_{12}, p_{21}, p_{22}$ are independent, identically distributed random variables with the standard normal distribution.
  \end{examples}

\end{frame}
%=========================================================================================
\begin{frame} \frametitle{Non-Homogenous Explicit $\D$-distributions}

  \begin{alertblock}{\deftitle{Diagonal Bands}}
    Suppose $P = (p_{ij})$ is an $N \times N$ matrix. Then, $P$ may be partitioned into $2n - 1$ rows called diagonal bands. Each band is denoted $[\rho]_P$ where $[\rho]_P = \{p_{ij} \mid \rho = i - j\}$. We have
    $\rho \in \{ -(N-1), \dots, -1, 0, 1, \dots, N-1 \}$.
  \end{alertblock}

\end{frame}
%=========================================================================================
\begin{frame} \frametitle{The Hermite $\beta$-Matrix}

  \begin{alertblock}{\deftitle{$\b$-matrix}}
    Suppose $P \sim \H(\b)$ is an $N \times N$ matrix. Then, the main diagonal $[0]_P \sim \Normal(0,2)$.
    Additionally, both the main off-digaonals are equal and they are given by $[1]_{P} = [-1]_{P} = \vec{X} = (X_k)_{k=1}^{N-1}$ where $X_k \sim \chi(\text{df} = \beta k)$.
    As such, we obtain a Hermite-$\b$ distributed matrix. Note that this is a symmetric tridiagonal matrix.
  \end{alertblock}

\end{frame}
%=========================================================================================
\begin{frame} \frametitle{Implicit Distributions}



\end{frame}
%=========================================================================================


%=========================================================================================
%   D-Distributions
%=========================================================================================
\section{Random Matrices}
%=========================================================================================
\begin{frame} \frametitle{Random Matrices}

\begin{alertblock}{Random Matrices}
Let $P \sim \D$ be an $N \times N$ matrix over $\F$. Then, the entries of $P$ are elements in $\F$ completely determined by the $\D$-distribution, regardless of what type it is.
Also, if $\D$ is an explicit distribution, $\D^\dagger$ represents the symmetric/hermitian version of $\D$.
\end{alertblock}
\end{frame}

%=========================================================================================
\begin{frame} \frametitle{Random Matrix Ensembles}

\begin{alertblock}{\deftitle{Random Matrix Ensembles}}
A $\D$-distributed ensemble $\Ens$ of $N \times N$ random matrices over $\F$ of size $K$ is defined as a set of $K$ iterations of that class of random matrix, and it is denoted:
$$ \Ens = \bigcup_{i = 1}^K P_i \where P_i \sim \mathcal{D} \and P_i \in \F^{N \times N} $$
\end{alertblock}

\end{frame}
%=========================================================================================
\begin{frame} \frametitle{Summary of $\D$-distributions}
  \begin{center}
    \Ddisttable
  \end{center}
\end{frame}

% %=========================================================================================
% \begin{frame}
% \frametitle{Empty Slide}
% \end{frame}
% %=========================================================================================
% \begin{frame}
% \frametitle{Empty Slide}
% \end{frame}


%=========================================================================================
%=========================================================================================
